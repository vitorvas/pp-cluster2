Computers have become a fundamental tool at almost every human activity.
In science, they're ubiquitous and used so widely that are often neglected
as a complex tool. A professional computer cluster is a good example of
this situation. Many high-end computers in expensive hardware configuration
put together using state-of-the-art software tools usually overlooked by
users as a 'set of computers'. In this paper we present such scientific tool
in a particular context: a high-end cluster system installed, managed and operated
by the very same scientific team which employs it on research on the field of Nuclear Engineering.
This work describes the main features of the LTHN/CDTN (Thermal-hydraulics and Neutronics
Laboratory/Nuclear Technology Development Center), considering software solutions
adopted, the impact of it on research currently carried on at the same laboratory
and the future possible applications of the system. A special attention is given to
the distributed file-system, a fundamental feature to allow high-performance in a system
with many physical disk storage units. All solutions adopted for the
system are based on open or free software and almost the totality of the research
software used is also open or free. Definitely the path towards less expensive
research, a theme of special interest for scientists and researches from developing countries.
